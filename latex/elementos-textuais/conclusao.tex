% ------------------------------------------------------------------------------
% Conclusão
% ------------------------------------------------------------------------------

\chapter{Conclusão}
\label{chap_conclusao}

O experimento mostrou que o algoritmo random walk utilizando um gerador de números aleatórios congruencial realmente gerou um caminhante aleatório. Conseguimos confirmar isso ao observar que o expoente da função da lei de potência ajustada tem expoente 1, indicando que a partícula em movimento não está confinada nem se movendo dentro de um fluxo de força externo.

O histograma desenhado com a posição final do caminhante mostra claramente que a distribuição de dados tem o formato de uma gaussiana. Ou seja, a distribuição de frequência da posição final do random walk respeita o teorema central do limite.

\section{Trabalhos futuros}
\label{sec_trabalhos_futuros}

Nesse trabalho exploramos o random walk apenas uma dimensão. Fazer simulações em mais dimensões e avaliar se as propriedades observadas serão as mesmas é uma maneira de dar continuidade a esse trabalho.  

% ------------------------------------------------------------------------------
% Observação: A norma ABNT estabelece que, em qualquer categoria de trabalho
% acadêmico monográfico deve haver um capítulo de conclusão
% ------------------------------------------------------------------------------
