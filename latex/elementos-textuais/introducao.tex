% ------------------------------------------------------------------------------
% Introdução
% ------------------------------------------------------------------------------

\chapter{Introdução}
\label{chap_introducao}

O Random walk, ou passeio aleatório em português, consiste em realizar sucessivos passos em direções aleatórias. O random walk pode ser realizado para qualquer número de dimensões e possui diversas aplicações práticas, como por exemplo o movimento browniano que descreve a movimentação de partículas suspensas em um fluido.

Neste trabalho criamos uma simulação de um random walk em uma dimensão, ou seja, a cada passo a entidade pode se mover para a direita ou para a esquerda e o módulo do passo é 1. Com a simulação podemos realizar análises sobre os dados e observar propriedades importantes do random walk.