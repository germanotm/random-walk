% ------------------------------------------------------------------------------
% Resumo
% ------------------------------------------------------------------------------

\begin{resumo}

    Simulação do ramdom walk ou caminhante aleatório em português, em uma dimensão. 
    A simulação foi realizada para 10 caminhantes realizando 10000 passos. Para 
    garantir a aleatoriedade do experimento foi utilizado um gerador de números 
    aleatórios congruencial. Os dados obtidos com as simulações foram utilizados para 
    desenhar gráficos log-log nos quais podemos observa a natureza da aleatoriedade do
    experimento. Aproximamos os dados obtidos com uma lei de potência para confirmar que
    se trata de um verdadeiro random walk. Também desenhamos um histograma com os resultados
    da simulação para mostrar que o random walk respeita o teorema central do limite.       

    \par\vspace{\baselineskip}

    \textbf{Palavras-chave}: Random Walk. Caminhante Aleatório. Gerador de Números Aleatórios Congruencial. Lei de potência. Teorema Central do limite.
\end{resumo}

% Para uma Tese de Doutorado o resumo deve conter, no máximo, 500 palavras.
% Para uma Dissertação de Mestrado o resumo deve conter, no máximo, 250 palavras.
% Para um Projeto de Qualificação o resumo deve conter, no máximo, 200 palavras.

% ------------------------------------------------------------------------------
% Escolha de 3 a 6 palavras ou termos que melhor representam seu trabalho.
% As palavras-chave são utilizadas para indexação. A letra inicial de cada
% palavra deve estar em maiúsculas. As palavras-chave são separadas por ponto.
% ------------------------------------------------------------------------------
